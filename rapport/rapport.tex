\documentclass[11pt, a4paper]{article}

\usepackage[francais]{babel}
\usepackage[utf8]{inputenc}
\usepackage[T1]{fontenc}

\usepackage{alltt}
\usepackage{amsmath}
\usepackage{amssymb}
\usepackage{mathabx}
\usepackage{vmargin}

\setmarginsrb{20mm}{20mm}{20mm}{20mm}{0mm}{0mm}{4mm}{8mm}

\renewcommand{\sfdefault}{phv}
\renewcommand{\rmdefault}{ptm}

% logique propositionnelle
\def\fm#1{\text{\texttt{#1}}}
\def\lequiv{\Leftrightarrow}
\def\limplique{\Rightarrow}
\def\louex{\oplus}

% logiques de descriptions
\def\SHOIND{{\mathcal{SHOIN}}(\fm{D})}
\def\SHOIQD{{\mathcal{SHOIQ}}(\fm{D})}
\def\roleR{\fm{r}}
\def\roleS{\fm{s}}
\def\instanceA{\fm{a}}
\def\instanceB{\fm{b}}
\def\conceptD{\fm{D}}
\def\conceptC{\fm{C}}
\def\conceptR{\fm{R}}
\def\attributG{g}
\def\subsumepar{\sqsubseteq}
\def\ET{\mathrel{\sqcap}}
\def\OU{\mathrel{\sqcup}}

% théorie des sous-ensembles flous
\def\Univers{{\mathcal{U}}}

% divers
\def\mlc#1{\text{\begin{tabular}{c}#1\end{tabular}}}
\def\mlg#1{\text{\begin{tabular}{l}#1\end{tabular}}}
\def\dans{\longrightarrow}
\def\Entiers{\mathbb{N}}

%%%%%%%%%%%%%%%%
\begin{document}
	%%%%%%%%%%%%%%%%
	
	%      ====================================
	\title{Exercice suivi du cours de C : E R O}
	%      ====================================
	\author{J. Lieber, Université de Lorraine, \texttt{Jean.Lieber@loria.fr}, dernière version : \today}
	\date{(Si vous voyez des erreurs, merci de me les signaler.)}
	\maketitle
	
	%         ----------------------
	\section*{\'Enoncé de l'exercice}
	%         ----------------------
	
	%          -----------
	\paragraph{Question 1.}
	%          -----------
	%
	Décrivez informellement un domaine de votre choix, dans lequel vous
	allez exprimer des connaissances.
	
	%          -----------
	\paragraph{Question 2.}
	%          -----------
	%
	Le formalisme considéré dans cette question est la logique propositionnelle.
	Dans ce formalisme, vous allez écrire, en alternance :
	\begin{itemize}
		\item
		Une phrase en français ou en anglais relative au domaine choisi ;
		\item
		Une représentation de cette phrase par une ou plusieurs formules de ce formalisme.
	\end{itemize}
	L'objectif étant d'utiliser à  bon escient tous les constructeurs de ce
	formalisme (pour la logique propositionnelle : les connecteurs logiques).
	
	Par ailleurs, vous aurez soin de choisir un vocabulaire explicite.
	Par exemple, pour exprimer le fait que les chats sont des félins,
	vous écrirez $\fm{chat}\limplique\fm{félin}$ plutôt que
	$c \limplique f$
	(même si vous expliquez préalablement que
	$c$ --- resp., $f$ --- représente le fait que
	l'individu considéré est un chat --- resp., un félin).
	
	Enfin, vous éviterez toutes les tautologies
	(p. ex., $\fm{chat}\lequiv\fm{chat}$,
	traduisant <<\,Un chat est un chat, et réciproquement.\,>>).
	
	%          -----------
	\paragraph{Question 3.}
	%          -----------
	Même question avec la logique du premier ordre.
	
	\paragraph{Question 4.}
	Même question avec RDFS.
	vous utiliserez au moins une fois chacune des propriétés
	suivantes :
	$\fm{rdf:type}$,
	$\fm{rdfs:subClassOf}$,
	$\fm{rdfs:subPropertyOf}$,
	$\fm{rdfs:domain}$ et
	$\fm{rdfs:range}$.
	
	%          -----------
	\paragraph{Question 5.}
	%          -----------
	Même question avec la logique de descriptions $\SHOIQD$.
	vous aurez soin d'utiliser des noms de concepts, de rôles et
	d'instances sous les bonnes casses :
	\begin{itemize}
		\item
		Un concept atomique commence par une majuscule et a une
		majuscule pour chaque <<\,nouveau mot\,>>, les autres
		lettres étant en minuscule
		(exemples : $\fm{Chat}$, $\fm{ChatNoirEtBlanc}$) ;
		\item
		Un nom de rôle commence par une minuscule et a une
		majuscule pour chaque <<\,nouveau mot\,>>, les autres
		lettres étant en minuscule
		(exemples : $\fm{aPourEnfant}$, $\fm{estUnEnfantDe}$),
		sachant que le premier mot contenu dans le nom du rôle est
		souvent un verbe à  la troisième personne du singulier de
		l'indicatif ;
		\item
		Un nom d'instance est en petites majuscules, la séparation
		entre mots se faisant par le caractère <<\,$\fm{\_}$\,>>
		(exemples : $\text{\textsc{gutenberg}}$,
		$\text{\textsc{igor\_stravinsky}}$).
	\end{itemize}
	
	%          -----------
	\paragraph{Question 6.}
	%          -----------
	%
	OWL~DL, dans sa version 1.0, est équivalent à  la logique de descriptions
	$\SHOIND$.
	Cependant, il existe plusieurs constructeurs de formules et de concepts
	propres à  OWL~DL qui ne sont pas dans $\SHOIND$ mais qui peuvent être
	traduits en $\SHOIND$.
	
	Le but de cette question est d'écrire des phrases et leurs traductions en
	OWL~DL (sous la syntaxe des logiques de descriptions) faisant appel aux
	constructeurs propres à OWL~DL suivants :
	$\ni\roleR.\instanceA$,
	<<\,$\roleR$ est fonctionnel\,>>,
	<<\,$\roleR$ est inverse fonctionnel\,>>,
	<<\,$\conceptD$ est domaine de $\roleR$\,>> et
	<<\,$\conceptR$ est co-domaine de $\roleR$\,>>.
	Enfin, vous donnerez la traduction en $\SHOIQD$ de ces formules.
	
	%          -----------
	\paragraph{Question 7.}
	%          -----------
	%
	Considérez une notion relative au domaine choisi qui, selon vous,
	<<\,relève du flou\,>> et modélisez cette notion à  l'aide d'un
	ensemble flou ou d'une relation floue.
	
	%          -----------
	\paragraph{Question 8.}
	%          -----------
	%
	Répondez à  la question~2 avec la logique propositionnelle
	possibiliste.
	
	%          -----------
	\paragraph{Question 9.}
	%          -----------
	%
	Même question avec une algèbre qualitative
	(soit l'algèbre de Allen, soit l'algèbre RCC8).
	
	%          ------------
	\paragraph{Question 10.}
	%          ------------
	%
	Décrivez un processus de changement de croyances
	(p. ex., révision ou contraction) dans le domaine choisi,
	en utilisant la logique propositionnelle comme formalisme
	pour représenter des croyances.
	
	Vous pourrez, par exemple, utiliser l'opérateur de révision
	de Dalal ou son opérateur de contraction.
	
	\strut\\
	\hrule
	\strut\\[-4mm]
	
	%         -----------------------------
	\section*{Question 1 : choix du domaine}
	%         -----------------------------
	
	Le domaine considéré est celui des rats-taupes glabres
	(leur organisation sociale, leur alimentation, etc.).
	
	%         ------------------------------------------------------
	\section*{Question 2 : formalisation en logique propositionnelle}
	%         ------------------------------------------------------
	
	\def\animalEusocial{\fm{animal\_eusocial}}
	\def\individuReproducteur{\fm{individu\_reproducteur}}
	\def\femelle{\fm{femelle}}
	\def\male{\fm{mâle}}
	\def\mammifere{\fm{mammifère}}
	\def\rongeur{\fm{rongeur}}
	\def\reineRTG{\fm{reine\_rat\_taupe\_glabre}}
	\def\RTD{\fm{rat\_taupe\_de\_Damara}}
	\def\RTG{\fm{rat\_taupe\_glabre}}
	
	Le rat-taupe glabre est un rongeur.
	\begin{equation*}
	\RTG \limplique \rongeur
	\tag{$\limplique$}
	\end{equation*}
	
	Les deux seules espèces de mammifère eusociaux sont
	le rat-taupe glabre et le rat-taupe de Damara\footnote{%
		En fait, ce sont les seules espèces \emph{connues} de mammifères eusociaux.}.
	\begin{equation*}
	(\mammifere \land \animalEusocial) \limplique (\RTG \lor \RTD)
	\tag{$\land$, $\lor$}
	\end{equation*}
	
	Ces deux espèces sont disjointes : il n'existe aucun individu à 
	la fois rat-taupe glabre et rat-taupe de Damara.
	\begin{equation*}
	\lnot\RTG
	\lor
	\lnot\RTD
	\tag{$\lnot$}
	\end{equation*}
	
	Une reine rat-taupe glabre est, par définition, une femelle reproductrice de cette espèce.
	\begin{equation*}
	\reineRTG \lequiv (\RTG \land \femelle \land \individuReproducteur)
	\tag{$\lequiv$}
	\end{equation*}
	
	Les rats-taupes glabres sont soit femelles, soit mâles.
	\begin{equation*}
	\RTG \limplique (\femelle \louex \male)
	\tag{$\louex$}
	\end{equation*}
	
	%         ------------------------------------------------------
	\section*{Question 3 : formalisation en logique du premier ordre}
	%         ------------------------------------------------------
	
	\bgroup
	\def\colonie{\fm{colonie}}
	\def\colonieRTG{\fm{colonie\_rat\_taupe\_glabre}}
	\def\estSitueEn{\fm{est\_situé\_en}}
	\def\ethiopie{\text{\textsc{éthiopie}}}
	\def\kenya{\text{\textsc{kenya}}}
	\def\laReineDeLaColonie{\fm{la\_reine\_de\_la\_colonie}}
	\def\reine{\fm{reine}}
	\def\RTG{\fm{rat\_taupe\_glabre}}
	\def\somalie{\text{\textsc{somalie}}}
	\def\vitDans{\fm{vit\_dans}}
	
	Tout rat-taupe glabre vit dans une colonie.
	\begin{equation*}
	\forall x ~ (\RTG(x) \limplique (\exists y ~ \vitDans(x, y) \land \colonie(y)))
	\tag{$\forall$, $\exists$, prédicats unaires et binaires}
	\end{equation*}
	
	Une colonie de rats-taupes glabres est, par définition, une colonie dans laquelle
	il existe au moins un rat-taupe glabre.
	\begin{equation*}
	\forall x ~
	(\colonieRTG(x) \lequiv
	\exists y ~ \vitDans(x, y) \land \RTG(y))
	\end{equation*}
	
	Dans toute colonie de rats-taupes glabres, il y a une reine (qui vit dans la colonie).
	\begin{equation*}
	\forall x ~
	(\colonieRTG(x) ~ \limplique \vitDans(x, \laReineDeLaColonie(x)))
	\tag{symboles de fonctions}
	\end{equation*}
	
	Les colonies de rats-taupes glabres se situent dans les trois pays suivants :
	l'éthiopie, le Kenya et la Somalie.
	\begin{equation*}
	\mlg{
		$\forall x ~
		(\colonieRTG(x) ~\limplique
		\left(
		\mlc{$\estSitueEn(x, \ethiopie)$ \\
			$\strut\lor\estSitueEn(x, \kenya)$ \\
			$\strut\lor\estSitueEn(x, \somalie)$}
		\right)$
		\\
		$\strut\land
		\exists x ~ \colonieRTG(x) \land \estSitueEn(x, \ethiopie)$
		\\
		$\strut\land
		\exists x ~ \colonieRTG(x) \land \estSitueEn(x, \kenya)$
		\\
		$\strut\land
		\exists x ~ \colonieRTG(x) \land \estSitueEn(x, \somalie)$
	}
	\tag{constantes}
	\end{equation*}
	\egroup
	
	%         ----------------------------------
	\section*{Question 4 : formalisation en RDfS}
	%         ----------------------------------
	
	Raymond est un rat-taupe glabre.
	\begin{equation*}
	(\fm{raymond} ~~ \fm{rdf:type} ~~ \fm{RatTaupeGlabre})
	\tag{$\fm{rdf:type}$}
	\end{equation*}
	
	Le rat-taupe glabre est un rongeur.
	\begin{equation*}
	(\fm{RatTaupeGlabre} ~~ \fm{rdfs:subClassOf} ~~ \fm{Rongeur})
	\tag{$\fm{rdfs:subClassOf}$}
	\end{equation*}
	
	Si $x$ est la reine de la colonie $y$ alors $x$ est membre de cette colonie.
	\begin{equation*}
	(\fm{estLaReineDe} ~~ \fm{rdfs:subPropertyOf} ~~ \fm{estMembreDe})
	\tag{$\fm{rdfs:subPropertyOf}$}
	\end{equation*}
	
	Si $x$ est la reine de $y$ alors $x$ est une reine
	et $y$ est une colonie.
	\begin{gather*}
	(\fm{estLaReineDe} ~~ \fm{rdfs:domain} ~~ \fm{Reine})
	\tag{$\fm{rdfs:domain}$}
	\\
	(\fm{estLaReineDe} ~~ \fm{rdfs:range} ~~ \fm{Colonie})
	\tag{$\fm{rdfs:range}$}
	\end{gather*}
	
	
	%         ---------------------------------------
	\section*{Question 5 : formalisation en $\SHOIQD$}
	%         ---------------------------------------
	
	\bgroup
	\def\alienor{\text{\textsc{aliénor}}}
	\def\aPourHabitant{\fm{aPourHabitant}}
	\def\aPourPetit{\fm{aPourPetit}}
	\def\aPourReine{\fm{aPourReine}}
	\def\aPourSexe{\fm{aPourSexe}}
	\def\estDansLaMemeColonieQue{\fm{estDansLaMêmeColonieQue}}
	\def\estSitueEn{\fm{estSituéEn}}
	\def\AnimalEusocial{\fm{AnimalEusocial}}
	\def\Colonie{\fm{Colonie}}
	\def\ColonieRTG{\fm{ColonieRTG}}
	\def\Mammifere{\fm{Mammifère}}
	\def\marcel{\text{\textsc{marcel}}}
	\def\Reine{\fm{Reine}}
	\def\RTG{\fm{RatTaupeGlabre}}
	\def\RTGMale{\fm{RatTaupeGlabreMâle}}
	\def\RTD{\fm{RatTaupeDeDamara}}
	\def\Rongeur{\fm{Rongeur}}
	\def\Sexefeminin{\fm{Sexeféminin}}
	\def\SexeMasculin{\fm{SexeMasculin}}
	\def\somalie{\text{\textsc{somalie}}}
	\def\tailleEnCentimetres{\fm{tailleEnCentimètres}}
	\def\vitDans{\fm{vitDans}}
	
	Le rat-taupe glabre est un rongeur.
	\begin{equation*}
	\RTG \subsumepar \Rongeur
	\tag{$\conceptC\subsumepar\conceptD$}
	\end{equation*}
	
	Les deux seules espèces de mammifère eusociaux sont le rat-taupe glabre et
	le rat-taupe de Damara.
	\begin{equation*}
	\Mammifere \ET \AnimalEusocial
	\subsumepar
	\RTG \OU \RTD
	\tag{$\ET$, $\OU$}
	\end{equation*}
	
	Ces deux espèces sont disjointes : il n'existe aucun individu à  la fois
	rat-taupe glabre et rat-taupe de Damara.
	\begin{equation*}
	\RTG \ET \RTD \subsumepar \bot
	\tag{$\bot$}
	\end{equation*}
	
	Un rat-taupe glabre de sexe féminin qui a eu un petit est nécessairement une reine.
	\begin{equation*}
	\RTG \ET \exists\aPourSexe.\Sexefeminin \ET \exists\aPourPetit.\top
	\subsumepar
	\Reine
	\tag{$\top$, $\exists\roleR.\conceptC$}
	\end{equation*}
	
	Un rat-taupe mâle est, par définition, un rat-taupe ayant un sexe masculin.
	\begin{equation*}
	\RTGMale
	\equiv
	\RTG \ET \exists\aPourSexe.\SexeMasculin
	\tag{$\equiv$}
	\end{equation*}
	
	Un rat-taupe qui n'est pas un mâle est un nécessairement de sexe féminin.
	\begin{equation*}
	\RTG \ET \lnot\RTGMale
	\subsumepar
	\exists\aPourSexe.\Sexefeminin
	\tag{$\lnot$}
	\end{equation*}
	
	Une colonie de rats-taupes glabres est, par définition,
	une colonie dans laquelle vit au moins
	un rat-taupe glabre et, dans une telle colonie,
	aucun rongeur d'une autre espèce ne vit.
	\begin{align*}
	\ColonieRTG \equiv \Colonie \ET \exists\aPourHabitant.\RTG
	\\
	\ColonieRTG \subsumepar \forall \aPourHabitant.(\lnot\Rongeur \OU \RTG)
	\tag{$\forall \roleR . \conceptC$}
	\end{align*}
	
	Marcel est un petit d'Aliénor, laquelle est une reine rat-taupe glabre.
	\begin{align*}
	&\aPourPetit(\alienor, \marcel)
	\tag{$\roleR(\instanceA, \instanceB)$}
	\\
	&(\Reine\ET\RTG)(\alienor)
	\tag{$\conceptC(\instanceA)$}
	\end{align*}
	
	Une colonie de rats-taupes glabres comprend entre $70$ et $300$ individus de cette espèce.
	\begin{equation*}
	\ColonieRTG \subsumepar
	(\geqslant 70 ~ \aPourHabitant . \RTG)
	\ET
	(\leqslant 300 ~ \aPourHabitant . \RTG)
	\tag{$(\geqslant n ~ \roleR . \conceptC)$, $(\leqslant n ~ \roleR . \conceptC)$}
	\end{equation*}
	
	La colonie de rats-taupes glabres dans laquelle vit la reine Aliénor
	contient exactement $200$ individus.
	\begin{equation*}
	(\ColonieRTG \ET \exists\aPourReine.\{\alienor\})
	\subsumepar
	(= 200 ~ \aPourHabitant . \RTG)
	\tag{$(= n ~ \roleR . \conceptC)$, $\{\instanceA_1, \ldots, \instanceA_n\}$}
	\end{equation*}
	(autrement écrit : toute colonie de rats-taupes glabres ayant pour
	reine Aliénor comprend $200$ individus).
	
	La taille d'un rat-taupe glabre est inférieure ou égale à  $33$ centimètres.
	\begin{equation*}
	\RTG
	\subsumepar
	\exists\tailleEnCentimetres.\leqslant_{\fm{33.}}
	\tag{$\exists\attributG.\varphi$}
	\end{equation*}
	
	Aliénor mesure $32$ centimètres.
	\begin{equation*}
	\tailleEnCentimetres(\alienor, \fm{32.})
	\end{equation*}
	
	Si $x$ est dans la même colonie que $y$ et que $y$ est dans la même colonie
	que $z$ alors $x$ est dans la même colonie que $z$.
	\begin{equation*}
	\text{<<\,$\estDansLaMemeColonieQue$ est transitif.\,>>}
	\tag{<<\,$\roleR$ est transitif.\,>>}
	\end{equation*}
	
	$x$ a pour habitant $y$ si et seulement si $y$ vit dans $x$.
	\begin{equation*}
	\aPourHabitant \equiv \vitDans^{-}
	\tag{$\roleS\equiv\roleR^{-}$}
	\end{equation*}
	
	La reine d'une colonie est un de ses habitants.
	\begin{equation*}
	\aPourReine \subsumepar \aPourHabitant
	\tag{$\roleR\subsumepar\roleS$}
	\end{equation*}
	
	%         ----------------------------------------
	\section*{Question 6 : formalisation en OWL~DL~1.0}
	%         ----------------------------------------
	
	La colonie dans laquelle vit Marcel est en Somalie
	(autrement écrit : l'ensemble des colonies où vit Marcel
	est inclus dans l'ensemble des colonies situées en Somalie).
	\begin{align*}
	&\ni\aPourHabitant.\marcel
	\subsumepar
	\ni\estSitueEn.\somalie
	\tag{$\ni\roleR.\instanceA$}
	\\
	&\exists\aPourHabitant.\{\marcel\}
	\subsumepar
	\exists\estSitueEn.\{\somalie\}
	\end{align*}
	
	Dans chaque colonie, il y a au plus une reine.
	\begin{align*}
	&\text{<<\,$\aPourReine$\,>> est fonctionnel.\,>>}
	\tag{<<\,$\roleR$ est fonctionnel.\,>>}
	\\
	&\top\subsumepar(\leqslant 1 ~ \aPourReine . \top)
	\end{align*}
	
	Une reine est reine d'une seule colonie.
	\begin{align*}
	&\text{<<\,$\aPourReine$\,>> est inverse fonctionnel.\,>>}
	\tag{<<\,$\roleR$ est inverse fonctionnel\,>>}
	\end{align*}
	
	Si $x$ a pour reine $y$ alors $x$ est une colonie.
	\begin{align*}
	&\text{<<\,$\Colonie$ est domaine de $\aPourReine$.\,>>}
	\tag{<<\,$\conceptD$ est domaine de $\roleR$.\,>>}
	\\
	&\exists\aPourReine.\top \subsumepar \Colonie
	\end{align*}
	
	Si $x$ a pour reine $y$ alors $y$ est une reine.
	\begin{align*}
	&\text{<<\,$\Reine$ est co-domaine de $\aPourReine$\,>>}
	\\
	&\top \subsumepar \forall \aPourReine.\Reine
	\end{align*}
	
	\egroup
	
	%         --------------------------------------------------------------------------
	\section*{Question 7 : modélisation par un sous-ensemble floue ou une relation floue}
	%         --------------------------------------------------------------------------
	
	\def\GCRTG{\fm{GCRTG}}
	\def\nbRTG{\fm{nb\_rtg}}
	
	La notion considérée est celle de <<\,grande colonie de rats-taupes glabres\,>>.
	
	Elle peut étre modélisée par un sous-ensemble flou de l'univers $\Univers$
	des colonies de rats-taupes glabres et à l'aide de la fonction
	$\nbRTG : \Univers \dans \Entiers$
	(où $\Entiers$ est l'ensemble des entiers naturels).
	
	La notion est définie par $\GCRTG$, sous-ensemble flou de $\Univers$, par :
	\begin{equation*}
	\text{pour $x\in\Univers$,}
	\qquad\qquad
	\mu_{\GCRTG}(x) =
	\begin{cases}
	0 & \text{si $\nbRTG(x) \leqslant 150$}
	\\
	\displaystyle\frac{\nbRTG(x)-150}{100}
	& \text{si $150 \leqslant \nbRTG(x) \leqslant 250$}
	\\
	1 & \text{si $\nbRTG(x) \geqslant 250$}
	\end{cases}
	\end{equation*}
	
	%         -------------------------------------------------------------------
	\section*{Question 8 : formalisation en logique propositionnelle possibiliste}
	%         -------------------------------------------------------------------
	
	Si un rat-taupe glabre n'est pas stérile, alors il est
	certain que c'est une reine ou un mâle :
	\begin{equation*}
	(\RTG \land 
	\individuReproducteur
	\limplique
	\reineRTG \lor \male,\quad
	\text{N}1)
	% \tag{N1}
	\end{equation*}
	
	Si $x$ est un humain et $y$ est un rat-taupe glabre,
	alors il n'est que faiblement possible que $x$ trouve $y$
	beau (ou belle), mais la réciproque, quoique possible, l'est encore
	moins.
	\begin{gather*}
	\text{avec }\varphi = \fm{humain} \land \fm{trouve\_que\_les\_rats\_taupes\_glabres\_sont\_beaux} \\
	\text{et } \psi = \RTG \land \fm{trouve\_que\_les\_humains\_sont\_beaux} \\
	(\varphi, \quad \Pi0{,}3) \qquad (\lnot\varphi, \quad \text{N}0{,}7) \\
	(\psi,    \quad \Pi0{,}2) \qquad (\lnot\psi,    \quad \text{N}0{,}8)
	\end{gather*}
	On remarquera dans cet exemple qu'un fragment de la logique
	possibiliste du premier ordre serait probablement plus approprié.
	
	%         --------------------------------------------------------------------
	\section*{Question 9 : formalisation dans l'algèbre de Allen ou l'algèbre RCC8}
	%         --------------------------------------------------------------------
	
	\def\Afrique{\fm{Afrique}}
	\def\CorneAfrique{\fm{CorneAfrique}}
	\def\HemisphereNord{\fm{HémisphèreNord}}
	\def\MerRouge{\fm{MerRouge}}
	\def\ZGRTG{\fm{ZG\_RTG}}
	
	L'algèbre RCC8 va nous permettre de représenter des connaissances relatives
	Ã  la zone géographique dans laquelle habite le rat-taupe glabre.
	
	La zone géographique du rat-taupe glabre ($\ZGRTG$) est une partie stricte de la Corne
	de l'Afrique :
	\begin{align*}
	&\ZGRTG \mathrel{\{TPP, NTPP\}} \CorneAfrique
	\\
	\text{ou, de façon équivalente : }
	&\CorneAfrique \mathrel{\{TPPi, NTPPi\}} \ZGRTG
	\end{align*}
	
	La corne de l'Afrique est située en partie dans l'hémisphère Nord :
	\begin{equation*}
	\CorneAfrique \mathrel{\{PO\}} \HemisphereNord
	\end{equation*}
	
	La corne de l'Afrique partage une frontière avec la mer Rouge :
	\begin{equation*}
	\CorneAfrique \mathrel{\{EC\}} \MerRouge
	\end{equation*}
	
	Les zones géographiques du rat-taupe glabre et de la taupe à  queue glabre
	sont disjointes :
	\begin{equation*}
	\ZGRTG \mathrel{\{DC\}} \fm{ZG\_taupe\_Ã \_queue\_glabre}
	\end{equation*}
	
	Et, afin d'avoir utilisé les $8$ relations d'RCC8, voici une
	tautologie :
	\begin{equation*}
	\ZGRTG \mathrel{\{=\}} \ZGRTG
	\end{equation*}
	
	%         ----------------------------------------------------
	\section*{Question 10 : changement de croyances sur le domaine}
	%         ----------------------------------------------------
	\def\rev{\dotplus}
	\def\contraction{\dotdiv}
	\def\dist{d_H}
	\def\F{\fm{F}}
	\def\V{\fm{V}}
	\def\interpretation{{\mathcal{I}}}
	\def\interpretationJ{{\mathcal{J}}}
	\def\I#1{\interpretation_{#1}}
	\def\Mod{{\mathcal{M}}}
	\def\poids#1{w_{#1}}
	
	Considérons les croyances suivantes :
	<<\,Les rats-taupes glabres sont des taupes.
	Les taupes sont des insectivores.\,>>
	On peut les formaliser comme suit :
	\begin{equation*}
	\psi = (\RTG \limplique \fm{taupe})
	\land (\fm{taupe} \limplique \fm{insectivore})
	\end{equation*}
	Or, suite à une discussion avec un autre agent,
	l'agent se met à douter du fait que les rats-taupes glabres
	sont insectivores.
	Il veut donc contracter $\psi$ par $\mu$, avec :
	\begin{equation*}
	\mu = \RTG \limplique \fm{insectivore}
	\end{equation*}
	
	Soit $\rev$ et $\contraction$, respectivement les opérateurs
	de révision et de contraction de Dalal, liés par l'identité de Harper :
	\begin{equation*}
	\psi \contraction \mu = \psi \lor (\psi \rev \lnot\mu)
	\end{equation*}
	Calculons $\psi\contraction\mu$.
	On notera $i=\fm{insectivore}$, $r=\RTG$
	et $t=\fm{taupe}$.
	On a le tableau suivant, dont les lignes représentent les interprétations
	sur les quatre variables propositionnelles considérées :
	{
		\begin{equation*}
		\begin{array}{| r || c | c | c || c  | c | c | c | c | c  |}
		\hline
		& i & r & t & \psi & \mu & \lnot\mu &
		\mlc{$\dist(\interpretation, \I3)$, \\ pour $\interpretation\models\psi$} &
		\mlc{$\dist(\interpretation, \I4)$, \\ pour $\interpretation\models\psi$} &
		\psi \contraction \mu
		\\
		\hline\hline
		\I1 & \F & \F & \F & \V & \V & \F & 1 & 2 & \V \\
		\I2 & \F & \F & \V & \F & \V & \F & - & - & \F \\
		\I3 & \F & \V & \F &\F & \F & \V & - & - & \V \\
		\I4 & \F & \V & \V & \F & \F & \V & - & - & \V \\
		\hline
		\I5 & \V & \F & \F & \V & \V & \F & 2 & 3 & \V \\
		\I6 & \V & \F & \V & \V & \V & \F & 3 & 2 & \V \\
		\I7 & \V & \V & \F & \F & \V & \F & - & - & \F \\
		\I8 & \V & \V & \V & \V & \V & \F & 2 & 1 & \V \\
		\hline
		\end{array}
		\end{equation*}
		
		Cherchons une formule équivalente à  $\psi\contraction\mu$ et
		facile à  interpréter.
		On a :
		\begin{align*}
		\Mod(\lnot(\psi\contraction\mu)) &= \{\I2, \I7\} \\
		\text{par conséquent, }
		\lnot(\psi\contraction\mu) &\equiv (\lnot{}i \land \lnot{}r \land t)
		\lor (i \land r \land \lnot{}t)
		\text{donc }
		\psi\contraction\mu &\equiv (i \lor r \lor \lnot{}t)
		\land (\lnot i \lor \lnot{}r \lor t) \\
		& \equiv (t \land \lnot{}r \limplique i)
		\land (i \land r \limplique t)
		\end{align*}
		La contraction a consisté Ã  affaiblir les deux implications de $\psi$
		en renforçant leurs parties gauches.
		
		%%%%%%%%%%%%%%
	\end{document}
	%%%%%%%%%%%%%%